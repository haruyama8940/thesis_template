\chapter{おわりに}
\label{chap:end}
\section{結論}
本論文では,
岡田らの従来手法に対し,動的に経路を選択して走行する機能や
視覚に基づいて通路の特徴を検出し,分岐路において目的地に向けた進行方向を提示する機能を追加することで,
走行する経路を一定の経路から,設定した任意の目的地に向けた経路へ拡張した.

はじめに経路選択機能の追加を目的として,データセットと学習器の入力へ目標方向を加えた.
そして,追加した機能の有効性をシミュレータを用いた実験により検証した.
実験では,視覚に基づくナビゲーションにおいて,
同一の分岐路であっても目標方向の入力に従い, 
ロボットが適切に経路を選択して移動する様子が見られた.

次に,視覚から通路の特徴を検出し,分岐路で目標方向を提示する機能を追加した.
目標方向を提示する機能には,島田らが提案したトポロジカルマップと
「条件」や「行動」による経路の表現(シナリオ)を用いた.
これにより,視覚に基づいて任意の目的地まで移動するシステムを構築した.
そして,実ロボットを用いた実験を行い,構築したシステムにより,カメラ画像とシナリオに基づいて, 
経路を追従し,ロボットが目的地へ到達できることを確認した.

% 事前に作成したメトリックマップを用いず, 
% カメラ画像とシナリオに基づいて経路を追従して目的地まで
% 自律移動するシステムを提案した.
% そして, 実ロボットを用いた実験により提案システムの有効性を検証した.
% 実験では, 提案システムにより,ロボットが目的地へ到達可能であることを
% 確認した.
% \section{今後の展望}
% 本論文では,津田沼キャンパス2号館3階の一部の廊下を対象として実験を行った.
% 今後の展望として,2号館3階のすべての廊下や屋外環境での実験が考えられる.