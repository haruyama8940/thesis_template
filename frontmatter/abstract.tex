%!TEX root = ../thesis.tex
\chapter*{概要}
\thispagestyle{empty}
%
\begin{center}
  \scalebox{1.5}{タイトル}\\
\end{center}
\vspace{1.0zh}
%


キーワード: \\
%
 実世界では,ある特定のセンサが機能しない状況におちいり.ロボットの自律移動が継続できない場合がある.
この問題に対しては,複数の種類のセンサを用いて自己位置推定する方法や,
ナビゲーション手段自体を冗長化する方法が考えられる.本研究グループでは,
冗長化に向けて,
一般的に用いられる LiDAR と地 図によるナビゲーションを機械学習で模倣することで,
視覚によるナビゲーションを獲得する方法を提案した [1][2].一般的な模倣学習が人の挙動を模倣するのに対 して,提案手法は LiDAR と地図によるナビゲーショ ンの出力を模倣するため,データセットを収集する手 間を省くことができるという特長がある.さらに前報 [1][2] では,分岐路で指定した方向 (以後, 目標方向と呼 ぶ) に移動する機能を追加した.これにより,ロボット は Fig.1 のように指示された方向に移動するように,カ メラ画像に基づいて経路を移動する.ただし, 前報まで のシステムは,目標方向をカメラ画像により生成して いなかったため, カメラ画像のみで目的地まで移動する ことはできなかった.
本稿では,カメラ画像のみで目的地に移動するため に,
カメラ画像から分岐路での目標方向を生成する機能を追加する.
具体的には,島田ら [3] が提案したトポ ロジカルマップと「条件」
や「行動」による経路の表 現(以後,シナリオと呼ぶ)をこれまで提案した手法
へ追加する.これにより,カメラ画像とトポロジカル マップから作成されるシナリオに基づいて,目的地ま で自律移動するシステムを構築する.このシステムに より,事前に作成したメトリックマップを必要せずに,
カメラ画像を入力として目的地まで自律移動できる可 能性がある.
メトリックマップを用いず,カメラ画像に基づいて 自律移動を行う研究はいくつかある.Dhruv ら [4] は 大規模な事前学習モデルを用いて,自然言語による指 示から,画像によるナビゲーションを end-to-end で行 う手法を提案している.また miyamoto ら [5] はカメラ 画像と深層学習による走行可能領域の検出とトポロジ カルマップを用いたナビゲーション手法を提案して いる.これらの手法では,補助的ではあるが,Global Navigation Satellite System(GNSS)やホイールオドメト リといった情報を必要としている.センサ入力という 観点で比較すると,本システムはカメラ画像のみで目 的地まで移動できるという違いがある.本稿では,提 案するシステムにより目的地までカメラ入力のみで自 律移動できるかを,実ロボットを用いた実験により検 証する.
\newpage
%%
\chapter*{abstract}
\thispagestyle{empty}
%
\begin{center}
  \scalebox{1.3}{title}
\end{center}
\vspace{1.0zh}
%


keywords: 
